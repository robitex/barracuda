% !TeX program = LuaTeX
% test ga-canvas
%
% Copyright (C) 2018 Roberto Giacomelli
%
\newbox\mybox

---ga--- is a binary format as an istruction sequence
that describes simple graphic object. This file contains
tests that aim to check the pdfliteral driver capability
to render ga streams---an usual Lua array.

The pdfliteral driver directly inserts PDF vector graphics
instruction within the output and should be intented as the
"native" driver.

To understand the format please read the "ga-grammar.pdf" file.
All dimensions are in scaled point, 65536sp = 1pt

Running the source file with luatex, the typesetting engine
executes the directlua macro. As a conseguence graphics
appear in the PDF output file.

Test number 1: a vbar 2pt width, 20pt height:
\directlua{
    pdfnat = require [[lib-driver.driver-pdfliteral]]
    pt = tex.sp [[1pt]] % pt = 65536
    % vbar: 36 y1 y2 nbars x1 w1 x2 w2 ... xn wn
    local ga = {36, 0, 20*pt, 1, 0.0, 2*pt}
    pdfnat:ga_to_hbox({_data = ga}, [[mybox]])
}\box\mybox

Test number 2: ten vbars equally spaced by 10pt:
\directlua{
    local ga = {36, 0, 10*pt, 10,}
    for i = 0, 9 do
        ga[i*2 + 5] = 5*pt + i*20*pt
        ga[i*2 + 6] = 10*pt
    end
    pdfnat:ga_to_hbox({_data = ga}, [[mybox]])
}\vrule\box\mybox\vrule

Test number 3: two series of vbars 10pt and 5pt large:
\directlua{
    local ga = {36, 0, 10*pt, 10,}
    for i = 0, 9 do
        ga[i*2 + 5] = 5*pt + i*20*pt
        ga[i*2 + 6] = 10*pt
    end
    ga[25] = 36 % vbar opcode
    ga[26] = 2.5*pt % y1
    ga[27] = 7.5*pt % y2
    ga[28] = 9      % number of bars
    for i = 0, 8 do
        ga[i*2 + 29] = 15*pt + i*20*pt
        ga[i*2 + 30] = 5*pt
    end
    pdfnat:ga_to_hbox({_data = ga}, [[mybox]])
}\vrule\box\mybox\vrule

Test number 4: a bunch of thin bars:
\directlua{
    local ga = {}
    ga[1] = 36    % vbar opcode
    ga[2] = 5*pt  % y1
    ga[3] = 25*pt % y2
    ga[4] = 25    % number of bars
    for i = 0, 24 do
        ga[i*2 + 5] = 1*pt + i*4*pt
        ga[i*2 + 6] = 2*pt
    end
    pdfnat:ga_to_hbox({_data = ga}, [[mybox]])
}\vrule\box\mybox\vrule

Test number 5: two floor of a bunch of thin bars:
\directlua{
    local ga = {}
    ga[1] = 36    % vbar opcode
    ga[2] = 5*pt  % y1
    ga[3] = 25*pt % y2
    ga[4] = 25    % number of bars
    for i = 0, 24 do
        ga[i*2 + 5] = 1*pt + i*4*pt
        ga[i*2 + 6] = 2*pt
    end
    ga[55] = 36    % vbar opcode
    ga[56] = 25*pt % y1
    ga[57] = 45*pt % y2
    ga[58] = 24    % number of bars
    for i = 0, 23 do
        ga[i*2 + 59] = 3*pt + i*4*pt
        ga[i*2 + 60] = 2*pt
    end
    pdfnat:ga_to_hbox({_data = ga}, [[mybox]])
}\vrule\box\mybox\vrule

Test number 6: staircase of bars (manual insertion of data):
\directlua{
    local ga = {}
    ga[1] =    36 % vbar opcode
    ga[2] =  0*pt % y1
    ga[3] = 20*pt % y2
    ga[4] =     1 % number of bars
    ga[5] =  5*pt % x
    ga[6] = 10*pt % w
    ga[7] =    36 % vbar opcode
    ga[8] = 20*pt % y1
    ga[9] = 40*pt % y2
    ga[10] =    1 % number of bars
    ga[11] = 15*pt % x
    ga[12] = 10*pt % w
    ga[13] =    36 % vbar opcode
    ga[14] = 40*pt % y1
    ga[15] = 60*pt % y2
    ga[16] =    1 % number of bars
    ga[17] = 25*pt % x
    ga[18] = 10*pt % w
    pdfnat:ga_to_hbox({_data = ga}, [[mybox]])
}\vrule\box\mybox\vrule


Test number 7: vbars with spaced text, all in three rows:
\directlua{
    local ga = {}
    ga[1] =    36 % vbar opcode
    ga[2] =  0*pt % y1
    ga[3] = 20*pt % y2
    ga[4] =     8 % number of bars
    for i = 1,8 do
        ga[3 + i*2] = i * 2 * 5*pt % x coordinate of bar axis
        ga[4 + i*2] = 5*pt % bar width
    end
    ga[21] =    36 % vbar opcode
    ga[22] = 30*pt % y1
    ga[23] = 50*pt % y2
    ga[24] =     8 % number of bars
    for i = 1,8 do
        ga[23 + i*2] = i * 2 * 5*pt % x coordinate of bar axis
        ga[24 + i*2] = 5*pt % bar width
    end
    % 131 <x1: FLOAT> <xgap: FLOAT> <ay: DIM> <ypos: DIM> <c: CHARS>
    ga[41] = 131   % opcode text_xspaced
    ga[42] = 10*pt % x coordinate of the first glyph axis
    ga[43] = 10*pt % x gap among glyphs
    ga[44] = 0.5   % half height
    ga[45] = 25*pt % y coordinate of glyps
    ga[46] = 65 % A
    ga[47] = 66 % B
    ga[48] = 67 % C
    ga[49] = 68 % D
    ga[50] = 69 % E
    ga[51] = 70 % F
    ga[52] = 71 % G
    ga[53] = 72 % H
    ga[54] = 0
    pdfnat:ga_to_hbox({_data = ga}, [[mybox]])
}\vrule\box\mybox\vrule

Test number 8: spaced text, check correct vertical alignment:
\directlua{
    local ga = {}
    ga[1] =    36 % vbar opcode
    ga[2] =  0*pt % y1
    ga[3] = 20*pt % y2
    ga[4] =     8 % number of bars
    for i = 1,8 do
        ga[3 + i*2] = i * 2 * 5*pt % x coordinate of bar axis
        ga[4 + i*2] = 1*pt % bar width
    end
    ga[21] =    36 % vbar opcode
    ga[22] = 40*pt % y1
    ga[23] = 60*pt % y2
    ga[24] =     8 % number of bars
    for i = 1,8 do
        ga[23 + i*2] = i * 2 * 5*pt % x coordinate of bar axis
        ga[24 + i*2] = 1*pt % bar width
    end
    % 131 <x1: FLOAT> <xgap: FLOAT> <ay: DIM> <ypos: DIM> <c: CHARS>
    ga[41] = 131   % opcode text_xspaced
    ga[42] = 10*pt % x coordinate of the first glyph axis
    ga[43] = 10*pt % x gap among glyphs
    ga[44] = 0.0   % half height
    ga[45] = 30*pt % y coordinate of glyps
    ga[46] = 65 % A
    ga[47] = 66 % B
    ga[48] = 67 % C
    ga[49] = string.byte("Q")
    ga[50] = 69 % E
    ga[51] = 70 % F
    ga[52] = 71 % G
    ga[53] = 72 % H
    ga[54] = 0
    ga[55] = 131   % opcode text_xspaced
    ga[56] = 10*pt % x coordinate of the first glyph axis
    ga[57] = 10*pt % x gap among glyphs
    ga[58] = 1.0   % half height
    ga[59] = 30*pt % y coordinate of glyps
    ga[60] = 49 % 1
    ga[61] = 50 % 2
    ga[62] = 51 % 3
    ga[63] = 52 % 4
    ga[64] = 53 % 5
    ga[65] = 54 % 6
    ga[66] = 55 % 7
    ga[67] = 56 % 8
    ga[68] = 0
    pdfnat:ga_to_hbox({_data = ga}, [[mybox]])
}\vrule\box\mybox\vrule

Test number 9: spaced text, check correct vertical alignment:
\directlua{
    local ga = {}
    ga[1] =    36 % vbar opcode
    ga[2] =  0*pt % y1
    ga[3] = 20*pt % y2
    ga[4] =     8 % number of bars
    for i = 1,8 do
        ga[3 + i*2] = i * 2 * 5*pt % x coordinate of bar axis
        ga[4 + i*2] = 8*pt % bar width
    end
    ga[21] =    36 % vbar opcode
    ga[22] = 40*pt % y1
    ga[23] = 60*pt % y2
    ga[24] =     8 % number of bars
    for i = 1,8 do
        ga[23 + i*2] = i * 2 * 5*pt % x coordinate of bar axis
        ga[24 + i*2] = 8*pt % bar width
    end
    % 131 <x1: FLOAT> <xgap: FLOAT> <ay: DIM> <ypos: DIM> <c: CHARS>
    ga[41] = 131   % opcode text_xspaced
    ga[42] = 10*pt % x coordinate of the first glyph axis
    ga[43] = 10*pt % x gap among glyphs
    ga[44] = 0.0   % half height
    ga[45] = 20*pt % y coordinate of glyps
    ga[46] = 65 % A
    ga[47] = 66 % B
    ga[48] = 67 % C
    ga[49] = 68 % D
    ga[50] = 69 % E
    ga[51] = 70 % F
    ga[52] = 71 % G
    ga[53] = 72 % H
    ga[54] = 0
    ga[55] = 131   % opcode text_xspaced
    ga[56] = 10*pt % x coordinate of the first glyph axis
    ga[57] = 10*pt % x gap among glyphs
    ga[58] = 1.0   % half height
    ga[59] = 40*pt % y coordinate of glyps
    ga[60] = 49 % 1
    ga[61] = 50 % 2
    ga[62] = 51 % 3
    ga[63] = 52 % 4
    ga[64] = 53 % 5
    ga[65] = 54 % 6
    ga[66] = 55 % 7
    ga[67] = 56 % 8
    ga[68] = 0
    pdfnat:ga_to_hbox({_data = ga}, [[mybox]])
}\vrule\box\mybox\vrule

Test number 10: two centered texts and baseline aligned:
\directlua{
    local ga = {}
    % 130 <ax: FLOAT> <ay: FLOAT> <xpos: DIM> <ypos: DIM> <c: CHARS>
    ga[ 1] = 130 % opcode text
    ga[ 2] = 0.5 % ax relative x coordinate
    ga[ 3] = 1.0 % ay relative y coordinate
    ga[ 4] = 0.0 % x position
    ga[ 5] = 0.0 % y position
    ga[ 6] =  65 % A
    ga[ 7] =  string.byte("Q") % Q depth glyph
    ga[ 8] =  67 % C
    ga[ 9] =   0
    ga[10] = 130 % opcode text
    ga[11] = 0.5 % ax
    ga[12] = 0.0 % ay
    ga[13] = 0.0 % x
    ga[14] = 0.0 % y
    ga[15] =  48 % 0
    ga[16] =  49 % 1
    ga[17] =  50 % 2
    ga[18] =  51 % 3
    ga[19] =  52 % 4
    ga[20] =  53 % 5
    ga[21] =  54 % 6
    ga[22] =  55 % 7
    ga[23] =  56 % 8
    ga[24] =  57 % 9
    ga[25] =   0
    pdfnat:ga_to_hbox({_data = ga}, [[mybox]])
}\vrule\box\mybox\vrule

So far, we have manually build data for a ga stream. This time we are going to
use ga-canvas library.

All the test are repeated 

Test 1: a vbar 2pt width, 20pt height:
\directlua{
    gacanvas = require "lib-geo.gacanvas"
    local ga = gacanvas:new()
    local err = ga:vbar(0.0, 0.0, 20*pt, {0.0, 2*pt}) % x, w
    assert(not err, err)
    pdfnat:ga_to_hbox(ga, [[mybox]])
}\box\mybox

Test 2: ten vbars equally spaced by 10pt:
\directlua{
    local ga = gacanvas:new()
    local bars = {}
    for i = 0, 9 do
        bars[i*2 + 1] =  5*pt + i*20*pt % x
        bars[i*2 + 2] = 10*pt           % w
    end
    local err = ga:vbar(0.0, 0.0, 10*pt, bars)
    assert(not err, err)
    pdfnat:ga_to_hbox(ga, [[mybox]])
}\vrule\box\mybox\vrule

Test 3: two series of vbars 10pt and 5pt large:
\directlua{
    local b1 = {}
    for i = 0, 9 do
        b1[i*2 + 1] = i*20*pt
        b1[i*2 + 2] = 10*pt
    end
    local b2 = {}
    for i = 0, 8 do
        b2[i*2 + 1] = i*20*pt
        b2[i*2 + 2] = 5*pt
    end
    local ga = gacanvas:new()
    local err = ga:vbar(0.0, 0.0, 10*pt, b1)
    assert(not err, err)
    err = ga:vbar(10.0*pt, 2.5*pt, 7.5*pt, b2)
    assert(not err, err)
    pdfnat:ga_to_hbox(ga, [[mybox]])
}\vrule\box\mybox\vrule

Test 4: a bunch of thin bars:
\directlua{
    local b = {}
    for i = 0, 24 do
        b[i*2 + 1] = 1*pt + i*4*pt
        b[i*2 + 2] = 2*pt
    end
    local ga = gacanvas:new()
    local err = ga:vbar(0.0, 5*pt, 25*pt, b)
    assert(not err, err)
    pdfnat:ga_to_hbox(ga, [[mybox]])
}\vrule{ }\box\mybox{ }\vrule

Test 5: two floor of a bunch of thin bars:
\directlua{
    local b = {}
    for i = 0, 24 do
        b[i*2 + 1] = i*4*pt
        b[i*2 + 2] = 2*pt
    end
    local ga = gacanvas:new()
    local err = ga:vbar(0.0, 5*pt, 25*pt, b)
    assert(not err, err)
    err = ga:vbar(2*pt, 25*pt, 45*pt, b)
    pdfnat:ga_to_hbox(ga, [[mybox]])
}\vrule\box\mybox\vrule

Test number 6: staircase of bars (manual insertion of data):
\directlua{
    local b = {0.0, 10*pt}
    local ga = gacanvas:new()
    local err = ga:vbar(0.0, 0.0, 20*pt, b)
    assert(not err, err)
    err = ga:vbar(10*pt, 20*pt, 40*pt, b)
    assert(not err, err)
    err = ga:vbar(20*pt, 40*pt, 60*pt, b)
    assert(not err, err)
    pdfnat:ga_to_hbox(ga, [[mybox]])
}\vrule\box\mybox\vrule

Test number 7: vbars with spaced text, all in three rows:
\directlua{
    local b = {}
    for i = 0,7 do
        b[i*2+1] = i*10*pt
        b[i*2+2] = 5*pt
    end
    local ga = gacanvas:new()
    local err = ga:vbar(0.0, 0.0, 20*pt, b)
    assert(not err, err)
    local err = ga:vbar(0.0, 30*pt, 50*pt, b)
    assert(not err, err)
    local c = {
        65, % A
        66, % B
        67, % C
        68, % D
        69, % E
        70, % F
        71, % G
        72, % H
    }
    err = ga:text_xspaced(0.0, 10*pt, 0.5, 25*pt, c)
    assert(not err, err)
    pdfnat:ga_to_hbox(ga, [[mybox]])
}\vrule\box\mybox\vrule

Test 8: spaced text, check correct vertical alignment:
\directlua{
    local b = {}
    for i = 0,7 do
        b[i*2+1] = i*10*pt
        b[i*2+2] = 2*pt
    end
    local ga = gacanvas:new()
    local err = ga:vbar(0.0, 0.0, 20*pt, b)
    assert(not err, err)
    local err = ga:vbar(0.0, 40*pt, 60*pt, b)
    assert(not err, err)
    local c = {
        65, % A
        66, % B
        67, % C
        string.byte("Q"),
        69, % E
        70, % F
        71, % G
        72, % H
    }
    err = ga:text_xspaced(0.0, 10*pt, 0.0, 30*pt, c)
    assert(not err, err)
    local n = {
        49, % 1
        50, % 2
        51, % 3
        52, % 4
        53, % 5
        54, % 6
        55, % 7
        56, % 8
    }
    err = ga:text_xspaced(0.0, 10*pt, 1.0, 30*pt, n)
    assert(not err, err)
    pdfnat:ga_to_hbox(ga, [[mybox]])
}\vrule\box\mybox\vrule

Test number 9: spaced text, check correct vertical alignment:
\directlua{
    local b = {}
    for i = 0,7 do
        b[i*2+1] = i*10*pt
        b[i*2+2] = 8*pt
    end
    local ga = gacanvas:new()
    local err = ga:vbar(0.0, 0.0, 20*pt, b)
    assert(not err, err)
    local err = ga:vbar(0.0, 40*pt, 60*pt, b)
    assert(not err, err)
    local c = {
        65, % A
        66, % B
        67, % C
        string.byte("Q"),
        69, % E
        70, % F
        71, % G
        72, % H
    }
    err = ga:text_xspaced(0.0, 10*pt, 0.0, 20*pt, c)
    assert(not err, err)
    local n = {
        49, % 1
        50, % 2
        51, % 3
        52, % 4
        53, % 5
        54, % 6
        55, % 7
        56, % 8
    }
    err = ga:text_xspaced(0.0, 10*pt, 1.0, 40*pt, n)
    assert(not err, err)
    pdfnat:ga_to_hbox(ga, [[mybox]])
}\vrule\box\mybox\vrule

Test number 10: two centered texts and baseline aligned:
\directlua{
    local n = {
        48, % 0
        49, % 1
        50, % 2
        51, % 3
        52, % 4
        53, % 5
        54, % 6
        55, % 7
        56, % 8
        57, % 9
    }
    local ga = gacanvas:new()
    local err = ga:text(0, 0, 0.5, 0, n)
    assert(not err, err)
    local a = {
        65, % A
        string.byte("Q"), % Q
        67, % C
    }
    err = ga:text(0, 0, 0.5, 1, a)
    assert(not err, err)
    pdfnat:ga_to_hbox(ga, [[mybox]])
}\vrule\box\mybox\vrule

Test number 11: two centered texts and baseline aligned:
\directlua{
    local n = {
        48, % 0
        49, % 1
        50, % 2
        51, % 3
        52, % 4
        53, % 5
        54, % 6
        55, % 7
        56, % 8
        57, % 9
    }
    local ga = gacanvas:new()
    local err = ga:text(0, 0, 0.5, 1, n)
    assert(not err, err)
    local a = {
        65, % A
        string.byte("Q"), % Q
        67, % C
    }
    err = ga:text(0, 0, 0.5, 0, a)
    assert(not err, err)
    pdfnat:ga_to_hbox(ga, [[mybox]])
}\vrule\box\mybox\vrule
\bye
