% !TeX program = LuaLaTeX

\documentclass{article}

\usepackage{fontspec}
\setmainfont{Libertinus Serif}

\usepackage[margin=18mm]{geometry}
\usepackage{booktabs}

\begin{document}
Goal: describe geometrical object like lines and rectangles
mainly for a barcode drawing library

\section{\texttt{ga} grammar}

A graphic data specification format called '\texttt{ga}' \emph{generic graphic assembler}.

\begin{verbatim}
ga<DIM, UINT> := generic graphic assembler
    <DIM>  := numeric type parameter for dimension, for example f64 or i32
    <UINT> := numeric type parameter for quantity, an unsigned integer, i.e. u8

ga<DIM, UINT> := +Elem<DIM, UINT>

Elem<DIM, UINT> := Code<u8> +Args<DIM, UINT>

Code<u8> := End<u8> | State<u8> | Object<u8> | Fn<u8>

End<u8>    :=   0        -- end sequence symbol (reserved for string serialization)
State<u8>  :=   1 ->  31 -- graphic properties
Object<u8> :=  32 -> 239 -- graphic object
Fn<u8>     := 240 -> 255 -- functions

Args<DIM, UINT> : <x: DIM> | <e: u8> | <n: UINT>
   <x: DIM> a dimension value of type DIM
   <e: u8> an enumeration value of type u8 (unsigned byte)
   <n: UINT> an unsigned integer for multiplicity
\end{verbatim}

\section{Properties}

Colors, linecap style etc\dots

\noindent\begin{tabular}{lll}
\toprule
Code/Mnemonic key & Graphic property & Operation\\
\midrule
\ttfamily 1 - pen\_thick & Line thick         & \ttfamily 1 <w: DIM>\\
\ttfamily 2 - pen\_cap\_style & Line cap style & \ttfamily 2 <e: u8>\\
\ttfamily 8 - color\\
\midrule
\ttfamily 30 - start\_bbox\_group & Stop to check the bounding box & \ttfamily 30\\
\ttfamily 31 - end\_bbox\_group & Pick bbox and restart to check & \ttfamily 31 <x1: DIM> <y1: DIM> <x2: DIM> <y2: DIM>\\
\bottomrule
\end{tabular}



\section{Objects}

\subsection{Lines}

A segment that starts from point P1 (x1, y1) and ends in P2 (x2, y2).

\noindent\begin{tabular}{lll}
\toprule
Code/Mnemonic key & Graphic object & Operation\\
\midrule
\ttfamily 32 - line & Line & \ttfamily 32 <x1: DIM> <y1: DIM> <x2: DIM> <y2: DIM>\\
\ttfamily 33 - line\_thick & Line with a thick & \ttfamily 33 <w: DIM> <x1: DIM> <y1: DIM> <x2: DIM> <y2: DIM>\\
\midrule
\ttfamily 36 - vbar & Vertical bars     & \ttfamily 36 <y1: DIM> <y2: DIM> <b: UINT> <x1: DIM> <t1: DIM> ...\\
\ttfamily 37 - hbar & Horizontal bars   & \ttfamily 37 <x1: DIM> <x2: DIM> <b: UINT> <y1: DIM> <t1: DIM> ...\\
\midrule
\ttfamily 38 - polyline & Open polyline     & \ttfamily 38 <n: UINT> <x1: DIM> <y1: DIM> <x2: DIM> <y2: DIM> ...\\
\ttfamily 39 - c\_polyline & Closed polyline   & \ttfamily 39 <n: UINT> <x1: DIM> <y1: DIM> <x2: DIM> <y2: DIM> ...\\
\bottomrule
\end{tabular}


\subsection{Rectangles}

\noindent\begin{tabular}{lll}
\toprule
Code/Mnemonic key & Graphic object & Operation\\
\midrule
\ttfamily 48 - rect & Rectangle & \ttfamily 48 <x1: DIM> <y1: DIM> <x2: DIM> <y2: DIM>\\
\ttfamily 49 - f\_rect & Filled rectangle & \ttfamily 49 <x1: DIM> <y1: DIM> <x2: DIM> <y2: DIM>\\
\ttfamily 50 - rect\_size & Rectangle & \ttfamily 50 <x1: DIM> <y1: DIM> <w: DIM> <h: DIM>\\
\ttfamily 51 - f\_rect\_size & Filled rectangle & \ttfamily 51 <x1: DIM> <y1: DIM> <w: DIM> <h: DIM>\\

\bottomrule
\end{tabular}


\subsection{Function}


\noindent\begin{tabular}{lll}
\toprule
Code/Mnemonic key & Function & Operation\\
\midrule
\ttfamily 240 - move & Translate objects & \ttfamily 240 <n: UINT> <dx: DIM> <dy: UINT>\\
\ttfamily 241 - copy & Copy object & \ttfamily 241 <n: UINT> <c: UINT> <dx1: DIM> <dy1: UINT> ...\\
\ttfamily 242 - and\_copy & Place and copy objects & \ttfamily 242 <n: UINT> <c: UINT> <dx1: DIM> <dy1: UINT> ...\\
\ttfamily 243 - grid & Copy next \(n\) objects on a grid & \ttfamily 243 <n: UINT> <col: UINT> <row: UINT> <dx: DIM> <dy: DIM>\\
\ttfamily 244 - sl\_grid\\
\ttfamily 250 - mirror\\
\ttfamily 255 - comment & A string comment & \ttfamily 255 <b1: u8> <b2: u8> .. 0\\
\bottomrule
\end{tabular}


\end{document}
