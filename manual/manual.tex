% !TeX program = ConTeXt

\starttext

\section{General framework}

The Barracuda package framework consists in different part:
a barcode class hierarchy encoding a text into a barcode symbology;
a geometrical library called libgeo representing several graphic object
an encoding library for the ga format (graphic assembler code)
several driver to "print" a ga streaming code into a file or TeX hbox register

To implement a barcode encoder you need to write a component called
builder producing the encoder class, while a driver must understand
ga opcode stream and print the corresponding graphic object.

Every barcode encoder come with a set of parameters, some of them can be
reserved and can be setting up by the user only through the builder.

So, you can create many instances of the same encoder for a single barcode
type, with its own parameter set.

The basic idea is getting faster encoder, for which the user may set up
paramenters at any level: barcode abstract class, encoder, down to a single
symbol.

Barcode class is completely indipendent from the ouput driver and viceversa.


\section{Reference}



\stoptext

